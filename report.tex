\documentclass{article}

\usepackage[top=1in, bottom=1in, left=1in, right=1in]{geometry}
\usepackage{amsmath, amssymb}
\usepackage{graphicx}
\usepackage{subcaption}
\graphicspath{ {/home/jackson/TDSErepo/figs/} }

\begin{document}

\title{Time Dependent Schr\"{o}dinger Equation}
\author{Jackson Burzynski, Caleb Helbling, Justin Lee}
\date{April 13, 2015}

\maketitle

\begin{abstract}

In this project, we have analyzed the solutions to the Time Dependent Sch\"{o}dinger Equation obtained through the finite difference scheme. Two different methods have been implemented and the results from each algorithm are compared. For simplicity, in this project we let $\hbar = m = \omega = 1$

\end{abstract}

\section{Free Particle}

\subsection{Comparing Methods}

In this project, we have implemented two different finite difference schemes. The first uses the "obvious" finite difference scheme below
%\begin{equation}
%\left[ \frac{\Psi_{j+1}^{n+1} -2 \Psi_j^{n+1} + \Psi_{j-1}^{n+1}}{(\Delta x)^2} \right]
%\end{equation}

The second uses a slightly modified approach. The second method is superior because it is stable and unitary. The algorithm for this method is

%\begin{equation}
%
%\end{equation}

\subsection{Periodic Boundary Conditions}

By applying periodic boundary conditions, we see the particle exit the frame on the right and reappear on the left, as expected

\subsection{Zero Boundary Conditions}

By applying zero boundary conditions, we effectively place the particle in the infinite square well. See the analysis in the following section. 

\section{Common Potentials}

\subsection{Infinite Square Well}

The eigenvalues for a particle in the infinite square well are

\begin{equation}
E_n = \frac{n^2 \pi^2 \hbar^2}{2 ma^2} = \frac{n^2 \pi ^2}{2 a^2}
\end{equation}
%
with corresponding eigenfunctions
\begin{equation}
\psi_n(x) = \sqrt{\frac{2}{a}} \cos \left( \frac{n\pi}{a} x \right)
\end{equation}

\subsection{Harmonic Oscillator}

The eigenvalues for a particle in this potential are

\begin{equation}
E_n \left( n + \frac{1}{2} \right) \hbar \omega = \left( n + \frac{1}{2} \right)
\end{equation}
%
with corresponding eigenfunctions

\begin{equation}
\psi_n(x) = \left( \frac{m \omega}{\pi \hbar} \right)^{1/4} \frac{1}{\sqrt{2^n n!}} H_n(x)e^{-x^2/2}
\end{equation}
%
where $H_n(x)$ is the n\textsuperscript{th}  Hermite Polynomial. The plots of the first two eigenstates are shown below for the two algorithms.
Note that the oscillations are periodic, as expected.

% insert diagrams here
\begin{figure}
\centering
\begin{subfigure}[t]{0.3\textwidth}
\centering
\includegraphics[width=2in]{img}
\caption{Algorithm 1}
\end{subfigure}
%
\begin{subfigure}[t]{0.3\textwidth}
\centering
\includegraphics[width=2in]{img}
\caption{Algorithm 2}
\end{subfigure}

\caption{Ground state of the harmonic oscillator}
\end{figure}

\section{Barrier Potential}

\subsection{Transmition and Reflection Coefficients}

\subsection{Incident Energy Equal to the Barrier Height}

\section{Kronig-Penney Crystal}

\section{Non-Hermition Hamiltonian}

We now look at the potential 

\[V(x) = \left\{
  \begin{array}{lr}
    ix &  : -L < x < < L \\
    \infty & : x \leq -L , x \geq L 
  \end{array}
\right.
\]

\end{document}
